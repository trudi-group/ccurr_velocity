
\begin{frame}{``Velocity'' of money}
  \begin{align*}%
    &\textbf{Before}\text{ the loss: } &\underbrace{2\,\coins}_{\text{\textbf{M}oney in circ.}} \, \cdot \, \underbrace{1}_{\text{Avg. turnovers}} = %
    \text{\textbf{T}ransaction volume} = 2\,\coins\\%
    &\textbf{After}\text{ the loss: }  &\underbrace{1\,\coins}_{\text{\textbf{M}oney in circ.}} \, \cdot \, \underbrace{2}_{\text{Avg. turnovers}} = %
    \text{\textbf{T}ransaction volume} = 2\,\coins%
  \end{align*} %
	%\vspace{1em}
	%\textbf{$\Longrightarrow$ }
	\begin{alertblockc}[N]{}{jwigreige!100!white}
		Velocity is the ``average number of turnovers during a period of time''.
	\end{alertblockc}
	%\vspace{1em}
	%\textbf{$\Longrightarrow\ $$\dots$ and calculated as:}
	\begin{align*}
	  \text{\textbf{V}elocity} = \frac{\text{\textbf{T}ransaction volume}}{\text{\textbf{M}oney in circulation}}
	\end{align*}
\end{frame}

\begin{frame}{``Velocity'' a bit more formal:}
	\centering
	\begin{align*}%
	  \Vp=\frac{
	  \overbrace{\langle\Pp,\Tp\rangle}^{\text{\textbf{T}ransaction volume (in coins)}}
	  }{
	  \underbrace{\Mp}_{\text{\textbf{M}oney in circulation (in coins)}}
	  }
	  \text{ with}\ \Mp, \Vp \in \mathbb{R}_{\ge 0},\ 
	  \text{and}\ \Pp,\Tp \in \mathbb{R}_{\ge 0}^{n}.%
	\end{align*}
	\vspace{1em}
	\begin{alertblockc}[]{}{jwigreige!100!white}
		\textbf{Velocity can be measured for UTXO-based cryptocurrencies like Bitcoin.}
	\end{alertblockc}
	\footnotetextNN{[Fisher, Irving: {T}he Equation of Exchange. 1911.]}
\end{frame}

\begin{frame}{Well now---which measures can we build on?}
	\begin{align*}
		\text{\textbf{V}elocity} = \frac{\text{\textbf{T}ransaction volume}}{\text{\textbf{M}oney in circulation}}
	\end{align*}
%	\vspace{1em}
	\begin{enumerate}\setlength{\itemsep}{0.5em}
		\item Just using \textbf{raw on-chain transaction volume} and \textbf{total coin supply}\\\deemphtext{---Literature: Bolt and Van Oord (2016), Ciaian et al. (2018)}%\cite{bolt2016value}, \cite{ciaian2018price}}
		\item Adjusting the on-chain transaction volume for \textbf{change transactions}\\\deemphtext{---Literature: Athey et al. (2016), Kalodner et al. (2017)}%\cite{kalodner2017blocksci}, \cite{athey2016bitcoin}} 
	\end{enumerate}
	\vspace{2em}
	\textit{"What is desired is the rate at which money is used for purchasing goods, not for making change."}---Fisher (1911)
\end{frame}

% \begin{frame}[t]{Empirical studies resort to approximations!}%
% 	\only<1>{%
% 		\begin{figure}%
% 			\centering%
% 			\includegraphics[scale=0.11]{./pics/used/paper-trail.png}%
% 		\end{figure}%
% 	}%
% 	\only<2>{%
% 		\begin{itemize}%
% 			% \item \cite{deleo2014does}% \visible<2->{\deemphtext{-- Yes, but negative!}}
% 			\item \cite{kancs2015digital}% \visible<2->{\deemphtext{-- Yes!}}
% 			\item \cite{georgoula2015using}% \visible<2->{\deemphtext{-- No.}}
% 			\item \cite{bouoiyour2015does}% \visible<2->{\deemphtext{-- No.}}
% 			\item \cite{ciaian2016digital}% \visible<2->{\deemphtext{-- Yes!}}
% 			\item \cite{ciaian2016economics}% \visible<2->{\deemphtext{-- No.}}
% 			\item \cite{luis2019drivers}% \visible<2->{\deemphtext{-- No!}}
% 			\item $\dots$%  
% 		\end{itemize}%
% 	}%
% %\visible<2->{\Large \centering{Hypothesis -- \emphtext{Proxy} -- Result}}
% \end{frame}%

% -------------------------------------------------

% \begin{frame}{Research Questions}

%   \begin{itemize}
%   \item Can we improve recently suggested \textbf{measures}? 
%   \item How well do simple \textbf{proxy-variables} used so far approximate the measures?
%   \end{itemize}

%   \vspace{2em}

%   \begin{block}{Teaser:}
%     We clean not only
%     \begin{itemize}
%     \item the transaction volume, but also 
%     \item the money supply
%     \end{itemize}
%     from artefacts.
%   \end{block}

% \end{frame}
